\documentclass[10pt]{article}
\usepackage[utf8]{inputenc}
\begin{document}

\section*{Ejercicio 5}

Estudiar la dependencia o independencia de las variables en cada una de las siguientes distribuciones. Dar, en cada caso, las curvas de regresión y la covarianza de las dos variables.

\begin{center}
\begin{tabular}{c|ccccc|c}
$X \backslash Y$ & 1 & 2 & 3 & 4 & 5 & $n_{i\cdot}$\\\hline
10 & 2 & 4 & 6 & 10 & 8 & 30\\
20 & 1 & 2 & 3 & 5 & 4 & 15\\
30 & 3 & 6 & 9 & 15 & 12 & 45\\
40 & 4 & 8 & 12 & 20 & 16 & 60\\\hline
$n_{\cdot j}$ & 10 & 20 & 30 & 50 & 40 & 150\\
\end{tabular}
\end{center}

Se aprecia que 

$$n_{ij} = \frac{n_{i\cdot}\cdot n_{\cdot j}}{n} \quad \forall j = 1,2,3,4,5 \quad \forall i = 1,2,3,4$$ 

por lo que son independientes.

Por ser independientes, la covarianza es nula: $$\sigma_{xy} = 0.$$

No tiene sentido en este caso hacer la recta y la curva de regresión. Aunque la recta de regresión sería $$y = 3,6$$ la curva de regresión de $X$ sobre $Y$ sería $$(29, 1), (29, 2), (29, 3), (29, 4), (29, 5)$$ y la curva de regresión de Y sobre X sería $$(10, 3.6), (20, 3.6), (30, 3.6), (40, 3.6), (50,3.6).$$

\bigskip

\begin{center}
\begin{tabular}{c|ccc|c}
$X \backslash Y$ & 1 & 2 & 3 & $n_{i\cdot}$\\\hline
-1 & 0 & 1 & 0 & 1 \\
0 & 1 & 0 & 1 & 2 \\
1 & 0 & 1 & 0 & 1 \\\hline
$n_{\cdot j}$ & 1 & 2 & 1 & 4 \\
\end{tabular}
\end{center}

En la segunda fila y en la segunda columna hay dos frecuencias no nulas, por lo que no hay dependencia funcional.

También, se aprecia que $$n_{11} = 0 \not = \frac{n_{1\cdot}\cdot n_{\cdot 1}}{n} = \frac{1}{4},$$ por lo que no son independientes.

$$m_{10} = \bar x = 0$$
$$m_{01} = \bar y = 2$$
$$m_{11} = 0$$
$$\sigma_{xy} = m_{11} - m_{10}m_{01} = 0$$

La covarianza es nula.

Podemos considerar una recta de regresión como $$y = 2.$$

La curva de regresión de $X$ sobre $Y$ es $$(0,1), (0,2), (0,3).$$

La curva de regresión de $Y$ sobre $X$ es $$(-1,2), (0,2), (1,2).$$

\end{document}