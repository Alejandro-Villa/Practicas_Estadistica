\documentclass[10pt]{article}
\usepackage{textcomp}
\usepackage[spanish]{babel}
\usepackage[utf8]{inputenc}
\begin{document}

\section*{Ejercicio 4}

Se realiza un estudio sobre la tensión de vapor de agua ($Y$, en ml. de Hg.) a distintas temperaturas($X$, en \textcelsius{}). Efectuadas 21 medidas, los resultados son:

\begin{center}
\begin{tabular}{c|ccc}
$X \backslash Y$ & (0.5, 1.5] & (1.5,2.5] & (2.5,5.5]\\\hline
(1, 15] & 1 & 2 & 0 \\
(15, 25] & 1 & 4 & 2 \\
(25, 30] & 0 & 3 & 5 \\
\end{tabular}
\end{center}

Explicar el comportamiento de la tensión de vapor en términos de la temperatura mediante una función lineal. ¿Es adecuado asumir este tipo de relación?

La recta de regresión es $$y = 0,0935x + 0,608.$$

En principio sí, porque se supone que a mayor temperatura habrá mayor evaporación.
Según el coeficiente de correlación lineal $$r = \frac{\sigma_{xy}}{\sigma_x\sigma_y} = 0,6565$$ el ajuste es más o menos adecuado, pero los hay mejores.


\end{document}
