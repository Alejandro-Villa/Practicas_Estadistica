\documentclass[10pt]{article}
\usepackage[spanish]{babel}
\usepackage[utf8]{inputenc}
\begin{document}

\section*{Ejercicio 3}

En una encuesta de familias sobre el número de individuos que la componen ($X$) y el número de personas activas en ellas ($Y$) se han obtenido los siguientes resultados:

\begin{center}
\begin{tabular}{c|cccc}
$X \backslash Y$ & 1 & 2 & 3 & 4 \\\hline
1 & 7 & 0 & 0 & 0 \\
2 & 10 & 2 & 0 & 0 \\
3 & 11 & 5 & 1 & 0 \\
4 & 10 & 6 & 6 & 0 \\
5 & 8 & 6 & 4 & 2 \\
6 & 1 & 2 & 3 & 1 \\
7 & 1 & 0 & 0 & 1 \\
8 & 0 & 0 & 1 & 1 \\
\end{tabular}
\end{center}

\begin{itemize}
\item[a)]Calcular la recta de regresión de $Y$ sobe $X$.

$y=0,3147x + 0,5322$.

\item[b)]¿Es adecuado suponer una relación lineal para explicar el comportamiento de $Y$ a partir de $X$?

Según el coeficiente de correlación lineal $$r = \frac{\sigma_{xy}}{\sigma_x\sigma_y} = 0,535$$ el ajuste no es el idóneo, pero los hay peores.

\end{itemize}
\end{document}
