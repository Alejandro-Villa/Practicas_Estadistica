\documentclass[12pt,a4paper]{article}
\usepackage[utf8]{inputenc}
\usepackage[spanish]{babel}
\usepackage{amsmath}
\usepackage{amsfonts}
\usepackage{amssymb}
\usepackage{enumerate}
\usepackage{graphicx}
\usepackage[left=2cm,right=2cm,top=2cm,bottom=2cm]{geometry}
\author{Alejandro Villanueva Prados}
\title{Relación 2 de EDIP.}
\begin{document}
\maketitle

\begin{enumerate}
        \item De las estadísticas de ``Tiempos de vuelo y consumos de combustible'' de una compañía aérea, se han obtenido datos relativos a 24 trayectos distintos realizados por eel avión DC-9. A partir de estos datos se han obtenido las siguientes medidas:
            \begin{center}
                \[
                \sum y_i = 219.719 \quad  \sum y_i^2 = 2396.504 \quad \sum x_i y_i = 349.486
                \]
                \[
                \sum x_i = 31.470 \quad \sum x_i^2 = 51.075 \quad \sum x_i^2 y_i = 633.993
                \]
                \[
                \sum x_i^4 = 182.977 \quad \sum x_i^3 = 93.6
                \]
            \end{center}
            La variable \(Y\) expresa el consumo total de combustible, en miles de libras, correspondiente a un vuelo de duración \(X\) (el tiempo se expresa en horas, y se utilizan como unidades de orden inferior fracciones decimales de la hora).

            \begin{enumerate}[a)]
                \item Ajustar un modelo del tipo \(Y=aX+b\). ¿Qué consumo total se estimaría para un programa de vuelos compuesto de 100 vuelos de media hora, 200 de una hora y 100 de dos horas? ¿Es fiable esta estimación?


                    \emph{Solución}: Comenzamos analizando nuestra población y los datos que tenemos: observamos que la población es de tamaño \(n=24\), además nos indican que los 24 vuelos son distintos así que \(n_i = 1; \quad \forall i \in \{1 \dots 24\}\) 


                    Una vez tomadas estas consideraciones, nos centramos en la pregunta: encontrar un ajuste lineal mediante un polinomio de grado 1. La expresión de la función será: \[y - \overline{y}=\frac{\sigma_{xy}}{\sigma_x^2}\left(x-\overline{x}\right),\] así que pasamos a calcular los datos necearios:\[\overline{x}=\frac{1}{n}\sum x_i = \frac{31.470}{24}\approx1.311\text{ Horas de vuelo},\]\[\overline{y}=\frac{1}{n}\sum y_i=\frac{219.719}{24}\approx 9.155\text{ Miles de libras de combustible},\]\[\sigma_{xy}=\frac{1}{n}\sum(x_i-\overline{x})(y_i-\overline{y})=m_{11}-m_{10}m_{01}=\frac{1}{n}\sum x_iy_i-\frac{1}{n^2}\sum x_i\sum y_i\approx 2.557\]\[\sigma_x^2=\frac{1}{n}\sum(x_i-\overline{x})^2=\frac{1}{n}\sum x_i^2 - \overline{x},\approx 0.410\] finalmente, la recta de regresión queda: \[y=6.237x-0.979.\]

                    Pasamos ahora a la segunda cuestión del apartado, para ello aplicamos la función obtenida:


                    \[ 6.237\cdot 0.5 -0.979 = 2.140 \text{ (Consumo estimado para un vuelo de media hora)}\]\[6.237 \cdot 1 - 0.979 = 5.258 \text{ (Consumo estimado para un vuelo de una hora)}\]\[6.237 \cdot 2 - 0.979 = 11.495 \text{ (Consumo estimado para un vuelo de dos horas)}\] Ahora escalamos estos resultados multiplicando por el númmero de vuelos y sumamos todo, obteniendo un consumo total de \(2415.1\) miles de libras de combustible.
                    
                    Para cuantificar la fiabilidad de la predicción, vamos a emplear el coeficiente de correlación lineal, dado por \[r=\pm\sqrt{r^2}=\frac{\sigma_{xy}}{\sigma_x\sigma_y},\] calculamos \(\sigma_x\) y \(\sigma_y\): \[\sigma_x=\sqrt{\sigma_x^2}=\sqrt{0.401}\approx0.640\]\[\sigma_y=\sqrt{\frac{1}{n}\left(\sum y_i^2-\sum y_i\right)}\approx 9.52,\] con estos datos, el coeficiente de correlación lineal queda \(0.42\), lo que nos indica un ajuste pobre y una baja fiabiidad.
                
                \item  Ajustar un modelo del tipo \(Y=a+bX+cX^2\). ¿Qué consumo total se estimaría para el mismo programa de vuelos del apartado a)?

                    \emph{Solución}: Los coeficientes de nuestra función de ajuste vienen dados por las soluciones del siguiente sistema:
%                    \begin{eqarray}
%                        m_{01} &= a_0 + a_1 m_{10} + a_2 m_{20}
%                    \end{eqarray}
            \end{enumerate}

\end{enumerate}

\end{document}
