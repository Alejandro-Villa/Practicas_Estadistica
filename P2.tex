\documentclass[12pt,a4paper]{article}
\usepackage[utf8]{inputenc}
\usepackage[spanish]{babel}
\usepackage{amsmath}
\usepackage{textcomp}
\usepackage{amsfonts}
\usepackage{amssymb}
\usepackage{enumerate}
\usepackage{graphicx}
\usepackage[left=2cm,right=2cm,top=2cm,bottom=2cm]{geometry}
\author{Gonzalo Moreno Soto\\ Alejandro Pérez Argüello\\ Miguel Piñar Pérez\\ Gerardo Tirado García\\ Irene Trigueros Lorca\\ Alejandro Villanueva Prados}
\title{Relación 2 de EDIP.}
\begin{document}
\maketitle

\begin{enumerate}
\item
\item
\item En una encuesta de familias sobre el número de individuos que la componen ($X$) y el número de personas activas en ellas ($Y$) se han obtenido los siguientes resultados:

\begin{center}
\begin{tabular}{c|cccc}
$X \backslash Y$ & 1 & 2 & 3 & 4 \\\hline
1 & 7 & 0 & 0 & 0 \\
2 & 10 & 2 & 0 & 0 \\
3 & 11 & 5 & 1 & 0 \\
4 & 10 & 6 & 6 & 0 \\
5 & 8 & 6 & 4 & 2 \\
6 & 1 & 2 & 3 & 1 \\
7 & 1 & 0 & 0 & 1 \\
8 & 0 & 0 & 1 & 1 \\
\end{tabular}
\end{center}

\begin{itemize}
\item[a)]Calcular la recta de regresión de $Y$ sobe $X$.

$y=0,3147x + 0,5322$.

\item[b)]¿Es adecuado suponer una relación lineal para explicar el comportamiento de $Y$ a partir de $X$?

Según el coeficiente de correlación lineal $$r = \frac{\sigma_{xy}}{\sigma_x\sigma_y} = 0,535$$ el ajuste no es el idóneo, pero los hay peores.

\end{itemize}
		\item Se realiza un estudio sobre la tensión de vapor de agua ($Y$, en ml. de Hg.) a distintas temperaturas($X$, en \textcelsius{}). Efectuadas 21 medidas, los resultados son:

\begin{center}
\begin{tabular}{c|ccc}
$X \backslash Y$ & (0.5, 1.5] & (1.5,2.5] & (2.5,5.5]\\\hline
(1, 15] & 1 & 2 & 0 \\
(15, 25] & 1 & 4 & 2 \\
(25, 30] & 0 & 3 & 5 \\
\end{tabular}
\end{center}

Explicar el comportamiento de la tensión de vapor en términos de la temperatura mediante una función lineal. ¿Es adecuado asumir este tipo de relación?

La recta de regresión es $$y = 0,0935x + 0,608.$$

En principio sí, porque se supone que a mayor temperatura habrá mayor evaporación.
Según el coeficiente de correlación lineal $$r = \frac{\sigma_{xy}}{\sigma_x\sigma_y} = 0,6565$$ el ajuste es más o menos adecuado, pero los hay mejores.

\item Estudiar la dependencia o independencia de las variables en cada una de las siguientes distribuciones. Dar, en cada caso, las curvas de regresión y la covarianza de las dos variables.

\begin{center}
\begin{tabular}{c|ccccc|c}
$X \backslash Y$ & 1 & 2 & 3 & 4 & 5 & $n_{i\cdot}$\\\hline
10 & 2 & 4 & 6 & 10 & 8 & 30\\
20 & 1 & 2 & 3 & 5 & 4 & 15\\
30 & 3 & 6 & 9 & 15 & 12 & 45\\
40 & 4 & 8 & 12 & 20 & 16 & 60\\\hline
$n_{\cdot j}$ & 10 & 20 & 30 & 50 & 40 & 150\\
\end{tabular}
\end{center}

Se aprecia que 

$$n_{ij} = \frac{n_{i\cdot}\cdot n_{\cdot j}}{n} \quad \forall j = 1,2,3,4,5 \quad \forall i = 1,2,3,4$$ 

por lo que son independientes.

Por ser independientes, la covarianza es nula: $$\sigma_{xy} = 0.$$

No tiene sentido en este caso hacer la recta y la curva de regresión. Aunque la recta de regresión sería $$y = 3,6$$ la curva de regresión de $X$ sobre $Y$ sería $$(29, 1), (29, 2), (29, 3), (29, 4), (29, 5)$$ y la curva de regresión de Y sobre X sería $$(10, 3.6), (20, 3.6), (30, 3.6), (40, 3.6), (50,3.6).$$

\bigskip

\begin{center}
\begin{tabular}{c|ccc|c}
$X \backslash Y$ & 1 & 2 & 3 & $n_{i\cdot}$\\\hline
-1 & 0 & 1 & 0 & 1 \\
0 & 1 & 0 & 1 & 2 \\
1 & 0 & 1 & 0 & 1 \\\hline
$n_{\cdot j}$ & 1 & 2 & 1 & 4 \\
\end{tabular}
\end{center}

En la segunda fila y en la segunda columna hay dos frecuencias no nulas, por lo que no hay dependencia funcional.

También, se aprecia que $$n_{11} = 0 \not = \frac{n_{1\cdot}\cdot n_{\cdot 1}}{n} = \frac{1}{4},$$ por lo que no son independientes.

$$m_{10} = \bar x = 0$$
$$m_{01} = \bar y = 2$$
$$m_{11} = 0$$
$$\sigma_{xy} = m_{11} - m_{10}m_{01} = 0$$

La covarianza es nula.

Podemos considerar una recta de regresión como $$y = 2.$$

La curva de regresión de $X$ sobre $Y$ es $$(0,1), (0,2), (0,3).$$

La curva de regresión de $Y$ sobre $X$ es $$(-1,2), (0,2), (1,2).$$

\item
\item
\item
\item
\item
\item


        \item De las estadísticas de ``Tiempos de vuelo y consumos de combustible'' de una compañía aérea, se han obtenido datos relativos a 24 trayectos distintos realizados por eel avión DC-9. A partir de estos datos se han obtenido las siguientes medidas:
            \begin{center}
                \[
                \sum y_i = 219.719 \quad  \sum y_i^2 = 2396.504 \quad \sum x_i y_i = 349.486
                \]
                \[
                \sum x_i = 31.470 \quad \sum x_i^2 = 51.075 \quad \sum x_i^2 y_i = 633.993
                \]
                \[
                \sum x_i^4 = 182.977 \quad \sum x_i^3 = 93.6
                \]
            \end{center}
            La variable \(Y\) expresa el consumo total de combustible, en miles de libras, correspondiente a un vuelo de duración \(X\) (el tiempo se expresa en horas, y se utilizan como unidades de orden inferior fracciones decimales de la hora).

            \begin{enumerate}[a)]
                \item Ajustar un modelo del tipo \(Y=aX+b\). ¿Qué consumo total se estimaría para un programa de vuelos compuesto de 100 vuelos de media hora, 200 de una hora y 100 de dos horas? ¿Es fiable esta estimación?


                    \emph{Solución}: Comenzamos analizando nuestra población y los datos que tenemos: observamos que la población es de tamaño \(n=24\), además nos indican que los 24 vuelos son distintos así que \(n_i = 1; \quad \forall i \in \{1 \dots 24\}\) 


                    Una vez tomadas estas consideraciones, nos centramos en la pregunta: encontrar un ajuste lineal mediante un polinomio de grado 1. La expresión de la función será: \[y - \overline{y}=\frac{\sigma_{xy}}{\sigma_x^2}\left(x-\overline{x}\right),\] así que pasamos a calcular los datos necearios:\[\overline{x}=\frac{1}{n}\sum x_i = \frac{31.470}{24}\approx1.311\text{ Horas de vuelo},\]\[\overline{y}=\frac{1}{n}\sum y_i=\frac{219.719}{24}\approx 9.155\text{ Miles de libras de combustible},\]\[\sigma_{xy}=\frac{1}{n}\sum(x_i-\overline{x})(y_i-\overline{y})=m_{11}-m_{10}m_{01}=\frac{1}{n}\sum x_iy_i-\frac{1}{n^2}\sum x_i\sum y_i\approx 2.557\]\[\sigma_x^2=\frac{1}{n}\sum(x_i-\overline{x})^2=\frac{1}{n}\sum x_i^2 - \overline{x}^2,\approx 0.40854\] finalmente, la recta de regresión queda: \[y=6.26x+0.946.\]

                    Pasamos ahora a la segunda cuestión del apartado, para ello aplicamos la función obtenida:


                    \[ 6.26\cdot 0.5+0.946 = 4.076 \text{ (Consumo estimado para un vuelo de media hora)}\]\[6.26 \cdot 1 + 0.946 = 7.206 \text{ (Consumo estimado para un vuelo de una hora)}\]\[6.26 \cdot 2 + 0.946 = 13.466 \text{ (Consumo estimado para un vuelo de dos horas)}\] Ahora escalamos estos resultados multiplicando por el númmero de vuelos y sumamos todo, obteniendo un consumo total de \(3195.4\) miles de libras de combustible.
                    
                    Para cuantificar la fiabilidad de la predicción, vamos a emplear el coeficiente de correlación lineal, dado por \[r=\pm\sqrt{r^2}=\frac{\sigma_{xy}}{\sigma_x\sigma_y},\] calculamos \(\sigma_x\) y \(\sigma_y\): \[\sigma_x=\sqrt{\sigma_x^2}=\sqrt{0.401}\approx0.640\]\[\sigma_y=\sqrt{\frac{1}{n}\left(\sum y_i^2-\frac{1}{n}\sum y_i\right)}\approx 4.005,\] con estos datos, el coeficiente de correlación lineal queda \(0.99758\), lo que nos indica un ajuste casi perfecto y una muy alta fiabiidad.
                
                \item  Ajustar un modelo del tipo \(Y=a+bX+cX^2\). ¿Qué consumo total se estimaría para el mismo programa de vuelos del apartado a)?

                    \emph{Solución}: Los coeficientes de nuestra función de ajuste vienen dados por las soluciones del siguiente sistema:
                    \begin{equation*}
                    \left\{ \begin{array}{rl}
                        m_{01} &= a_0 + a_1 m_{10} + a_2 m_{20}, \\
                        m_{11} &= a_0 m_{10} + a_1 m_{20} + a_2 m_{30}, \\
                        m_{21} &= a_0 m_{20} + a_1 m_{30} + a_2 m_{40}.                    
                    \end{array}\right.
                    \end{equation*}
Para ello calculamos los distintos momentos valiéndonos de los datos iniciales y resolvemos el sistema, obtenemos:
\[
a_0 = 0.800 \quad a_1 = 6.558 \quad a_2 = -0.112.
\]

Con estos resultados nuestra función de ajuste nos queda: \(Y = -0.112X^2 + 6.558X + 0.8\). Volviendo a calcular las predicciones con esta función nos queda un consumo estimado de \(3198.275\) miles de libras de combustible. 
\item ¿Cuál de los dos modelos se ajusta mejor? Razonar la respuesta.

\emph{Solución}: Para comparar los dos modelos vamos a usar el coeficiente de correlación, definido como sigue:
\[
\eta^2\_{Y/X}=\frac{\sigma_{ey}^2}{\sigma_y^2},
\]
donde \(\sigma_{ey}^2 = \frac{1}{n}\sum \left( \hat{y}_j-\overline{y}\right)^2=\)

\textbf{NOTA: no está terminado el apartado c), no sé cómo hacerlo :'(}
\end{enumerate}

\item
\item


\end{enumerate}

\end{document}